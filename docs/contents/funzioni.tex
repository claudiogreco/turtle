\section{Funzioni}

\subsection{NumberType}

Ciascun termine delle seguenti funzioni deve essere numerico (\verb!IntegerType! o \verb!FloatType!) oppure una variabile di tipo numerico che verrà sostituita automaticamente se istanziata correttamente.

\subsubsection{Addition}
\emph{Sintassi}: ($+$ $\mathrm{x_1 \ x_2 \ ... \ x_n}$)\\

Somma i termini $x_1, ..., x_n$. Il passo base è \verb!IntgerType(0)!.


\subsubsection{Subtraction}
\emph{Sintassi}: ($-$ $\mathrm{x_1 \ x_2 \ ... \ x_n}$)\\

Sottrae i termini $x_1, ..., x_n$. Il passo base è $x_n$.


\subsubsection{Multiplication}
\emph{Sintassi}: ($*$ $ \mathrm{x_1 \ x_2 \ ... \ x_n}$)\\

Moltiplica i termini $x_1, ..., x_n$. Il passo base è \verb!IntegerType(1)!.


\subsubsection{Division}
\emph{Sintassi}: ($/$ $ \mathrm{x_1 \ x_2 \ ... \ x_n}$)\\

Divide i termini $x_1, ..., x_n$. Il passo base è $x_1$.


\subsubsection{Module}
\emph{Sintassi}: ($\%$ $ \mathrm{x_1 \ x_2 \ ... \ x_n}$)\\

Calcola il resto della divisione tra i termini $x_1, ..., x_n$. Il passo base è $x_1$.


\subsubsection{Power}
\emph{Sintassi}: ($**$ $ \mathrm{x_1 \ x_2 \ ... \ x_n}$)\\

Calcola la potenza di $x_1$ con esponente il prodotto dei termini $x_2, ..., x_n$. Il passo base è $x_1$.


\subsubsection{Abs}
\emph{Sintassi}: ($abs$ $ \mathrm{x}$)\\

Calcola il valore assoluto di $x$. E' un wrapper per la funzione \verb!abs! di Python.


\subsubsection{Minimum}
\emph{Sintassi}: ($min$ $ \mathrm{x_1 \ x_2 \ ... \ x_n}$)\\

Trova il minimo tra i termini $x_1, ..., x_n$. E' un wrapper per la \verb!min! di Python.


\subsubsection{Maximum}
\emph{Sintassi}: ($max$ $ \mathrm{x_1 \ x_2 \ ... \ x_n}$)\\

Trova il massimo tra i termini $x_1, ..., x_n$. E' un wrapper per la funzione \verb!max! di Python.

\subsubsection{Randint}
\emph{Sintassi}: ($randint$ $ \mathrm{x_1 \ x_2}$)\\

Restituisce un numero intero casuale compreso tra $x_1$ e $x_2$ inclusi. Il primo termine non deve essere necessariamente più piccolo del secondo.

\subsection{LexemeType: StringType}

\subsubsection{Strcat}
\emph{Sintassi}: ($strcat$ $ \mathrm{x_1 \ x_2 \ ... \ x_n}$)\\

Concatena le stringhe $x_1, ..., x_n$ restituendo un nuovo  \verb!StringType!.

\subsubsection{Substr}
\emph{Sintassi}: ($substr$ $ \mathrm{string \ start \ end}$)\\

Restituisce una sottostringa compresa tra gli indici di posizione start ed end (escluso) di una stringa passata in input. La funzione si comporta come lo \emph{slicing} di Python, in cui l'intervallo considerato inizia dalla posizione start e termina nella posizione end-1. Se entrambi gli indici superano il massimo range specificabile, viene restituita l'intera stringa in input. Se uno solo degli indici supera il range di posizioni possibili si hanno due casi: se \verb!start! è out of range allora viene restituita una sottostringa data dagli indici \verb!0! ed \verb!end! (\verb![0:end]!); se \verb!end! è  out of range viene restituita una sottostringa data dagli indici \verb!start! e \verb!ultima posizione stringa in input! (\verb![start:]!).

\subsubsection{Strlen}
\emph{Sintassi}: ($strlen$ $ \mathrm{string}$)\\

Calcola la lunghezza della stringa in input ed utilizza un \verb!IntegerType! come valore di ritorno.

\subsubsection{Strindex}
\emph{Sintassi}: ($strindex$ $ \mathrm{string \ substring}$)\\

Cerca un carattere o una sottostringa all'interno di una stringa e, in caso di successo, restituisce la posizione iniziale della sequenza di caratteri ricercata. In caso contrario viene restituito l'intero negativo \verb!-1!. Il risultato viene memorizzato in un \verb!IntegerType! utilizzato come valore di ritorno.

\subsection{LexemeType: SymbolType}

\subsubsection{Symcat}
\emph{Sintassi}: ($symcat$ $ \mathrm{x_1 \ x_2 \ ... \ x_n}$)\\

Concatena i simboli $x_1, ..., x_n$ restituendo un nuovo  \verb!SymbolType!.

